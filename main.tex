\documentclass{article}
\usepackage[utf8]{inputenc}
\usepackage[letterpaper, margin=1in]{geometry}
\usepackage{longtable}
\usepackage{multirow}

\title{MUPRO-Ferroelectrics}
\author{bzw133 }
\date{April 2017}

\begin{document}

\maketitle

\clearpage{}

{
\centering
\tableofcontents
}

\clearpage{}

\section{MuFerro: an introduction}
\subsection{Structure of the program}
\subsection{Features}
\subsection{A quick tutorial}

\section{Installation of MuFerro}

\section{Files used by MuFerro}
\subsection{Input files}
\subsubsection{Input.in}
\subsubsection{Pot.in}
\subsubsection{Polar.in} 
\subsubsection{OctaTilt.in}
\subsubsection{AFMtip.in}
\subsection{Output files}
\subsection{Other files}
\subsubsection{pbs file for submitting tasks}
\subsubsection{scripts for batch tasks}
\subsubsection{scripts for postprocessing}

\section{The Input.in File}

\subsection{List of most important parameters}

%\begin{table}[]
%\centering
%\caption{My caption}
%\label{my-label}
%\begin{tabular}{|c|l|l|l|l|}

\begin{longtable}{|p{0.15\textwidth}|p{0.15\textwidth}|p{0.1\textwidth}|p{0.3\textwidth}|p{0.25\textwidth}|}
\caption{Your caption here} % needs to go inside longtable environment
\label{tab:myfirstlongtable}

\hline
 & \textbf{Tags} & \textbf{Class} & \multicolumn{1}{c}{\textbf{Default}} & \multicolumn{1}{c}{\textbf{Description}} \\ \hline
\multirow{5}{*}{\textbf{System}} & SYSDIM & Vector & 10, 10, 10 & Grid points in x, y, z \\ \cline{2-5} 
 & REALDIM & Vector & Same as SYSDIM & Real length along x, y, z \\ \cline{2-5} 
 & SUBTHICK & Scalar & 0 & Grid points for substrate thickness \\ \cline{2-5} 
 & FILMTHICK & Vector & 0 & Grid points for film thickness \\ \cline{2-5} 
 & ROTANGLE & Vector & 0.0, 0.0, 0.0 & Euler rotation angle around z, x', z'' \\ \hline
\multirow{4}{*}{\textbf{Time}} & TTOTAL & Scalar & 1000 & Total timesteps \\ \cline{2-5} 
 & TSTART & Scalar & 0 & Initial timestep \\ \cline{2-5} 
 & TOUT & Scalar & 200 & Time interval for output \\ \cline{2-5} 
 & TDELTA & Scalar & 0.01 & Delta t of TDGL \\ \hline
\multirow{2}{*}{\textbf{Solvers}} & LINHOM & Flag & FALSE & Flag for inhomogeneous solver \\ \cline{2-5} 
 & CDER & Choice & \begin{tabular}[c]{@{}l@{}}0- step-corrected FFT\\ 1- FDM\\ 2- normal FFT\end{tabular} & Choices of derivative solver \\ \hline
\textbf{Thermal} & TEM & Scalar & 298.0 & Temperature in Kelvin \\ \hline
\multirow{5}{*}{\textbf{Polarization}} & CPOLARBC & Choice & \begin{tabular}[c]{@{}l@{}}0- natural BC\\ 1- free BC\\ 2- blocked BC\\ 3- small extrapolation length\\ 4- zero bound charge\\ 5- large extrapolation length\end{tabular} & Choices of boundary condition,for Px, Py, Pz \\ \cline{2-5} 
 & ETRLP & Vector & 0.0, 0.0, 0.0 & Extrapolation length of Px, Py, Pz in nm \\ \cline{2-5} 
 & LPNOISE & Flag & FALSE & Flag of polarization noise \\ \cline{2-5} 
 & PNOISMAG & Scalar & 0.1 & Magnitude of polarization noise \\ \cline{2-5} 
 & PNOISEED & Scalar & 10 & Seed of polarization noise \\ \hline
\multirow{6}{*}{\textbf{Elastic}} & LELAS & Flag & TRUE & Flag of elastic energy \\ \cline{2-5} 
 & CELASBC & Choice & \begin{tabular}[c]{@{}l@{}}0- film\\ 1- bulk strain\\ 2- bulk stress\end{tabular} & Choices of elastic boundary conditions \\ \cline{2-5} 
 & MISFIT & Vector & 0.0, 0.0, 0.0 & Misfit strain exx, eyy, exy \\ \cline{2-5} 
 & STRAIN & Vector & 0.0, 0.0, 0.0, 0.0, 0.0, 0.0 & Strain BC for bulk, exx, eyy, ezz, eyz, ezx, exy \\ \cline{2-5} 
 & STRESS & Vector & 0.0, 0.0, 0.0, 0.0, 0.0, 0.0 & Stress BC for bulk, sxx, syy, szz, syz, szx, sxy \\ \cline{2-5} 
 & TIPRAD & Scalar & 50.0 & Tip radius in nm \\ \hline
\multirow{5}{*}{\textbf{Electric}} & LELEC & Flag & TRUE & Flag of electric energy \\ \cline{2-5} 
 & CELECBC & Choice & \begin{tabular}[c]{@{}l@{}}1- open circuit\\ 2- short circuit\\ 3- top open bottom short\\ 4- top short bottom open\\ 5- bulk\end{tabular} & Choices of electric boundary conditions \\ \cline{2-5} 
 & TIPGAMMA & Scalar & 10.0 & Lortenz γ for tip in nm \\ \cline{2-5} 
 & SCRBOT & Scalar & 0.0 & Screening factor of bottom electrode \\ \cline{2-5} 
 & SCRTOP & Scalar & 0.0 & Screening factor of top electrode \\ \hline
\textbf{Gradient} & GRADCON & Vector & 0.6, 0.0, 0.3 & Gradient energy tensor component g11, g12 and g44 \\ \hline
\multirow{2}{*}{\textbf{Flexoelectric}} & LFLEXO & Flag & FALSE & Flag of flexoelectric energy \\ \cline{2-5} 
 & FLEXOCON & Vector & 5.1, 3.3, 0.045 & Flexoelectric tensor component f11, f12, and f44 \\ \hline
\multirow{5}{*}{\textbf{\begin{tabular}[c]{@{}c@{}}Oxygen \\ Octahedral \\ Rotation\end{tabular}}} & LOCTILT & Flag & FALSE & Flag of oxygen octahedral tilt \\ \cline{2-5} 
 & GRADQCON & Vector & 1.0 0.0 0.5 & Gradient energy tensor of OOT component v11, v12, v44 \\ \cline{2-5} 
 & LQNOISE & Flag & FALSE & Flag of OOT noise \\ \cline{2-5} 
 & QNOISMAG & Scalar & 0.1 & Magnitude of OOT noise \\ \cline{2-5} 
 & QNOISEED & Scalar & 10 & Seed of  OOT noise \\ \hline
\multirow{6}{*}{\textbf{Outputs}} & LOUTLAND & Flag & FALSE & Flag of output Landau energy of polarization \\ \cline{2-5} 
 & LOUTELAS & Flag & FALSE & Flag of output elastic energy \\ \cline{2-5} 
 & LOUTELEC & Flag & FALSE & Flag of output elastic energy \\ \cline{2-5} 
 & LOUTGRAD & Flag & FALSE & Flag of output gradient energy \\ \cline{2-5} 
 & LOUTFLEX & Flag & FALSE & Flag of output flexoelectric fields \\ \cline{2-5} 
 & LOUTFORC & Flag & FALSE & Flag of output driving forces \\ \hline

\end{longtable}
%\end{table}


\subsection{CDER     }

\subsection{CELASBC  }

\subsection{CELECBC  }

\subsection{CPOLARBC }

\subsection{DIELECON }

\subsection{ELECFIELD}

\subsection{ETRLP    }

\subsection{FILMTHICK}

\subsection{FLEXOCON }

\subsection{GRADPCON }

\subsection{GRADQCON }

\subsection{LAFMTIP  }

\subsection{LELAS    }

\subsection{LELEC    }

\subsection{LFLEXO   }

\subsection{LINHOM   }
xx
\subsection{LOCTILT  }

\subsection{LOUTELAS }

\subsection{LOUTELEC }

\subsection{LOUTFLEX }

\subsection{LOUTFORC }

\subsection{LOUTGRAD }

\subsection{LOUTLAND }

\subsection{LPNOISE  }

\subsection{LQNOISE  }

\subsection{MISFIT   }

\subsection{PNOISEED }

\subsection{PNOISMAG }


\subsection{QNOISEED }

\subsection{QNOISMAG }

\subsection{REALDIM  }

\subsection{ROTANGLE }

\subsection{SCREENTOP}

\subsection{SCREENBOT}

\subsection{STRAIN   }

\subsection{STRESS   }

\subsection{SUBTHICK }

\subsection{SYSDIM   }

\subsection{TDELTA   }

\subsection{TEM      }

\subsection{TIPGAMMA }

\subsection{TIPRAD   }

\subsection{TOUTPUT  }

\subsection{TSTART   }

\subsection{TTOTAL   }

\clearpage{}
\section{Theoretical Background}

\subsection{Ginzburg-Landau-Devonshire (LGD) theory of ferroelectrics}
This section describes the LGD theory of ferroelectrics briefly

\subsection{Phase-field methods of ferroelectrics}
This section describes the phase field method of ferroelectrics briefly 

\subsection{Semi-implicit Fourier methods}
This section shortly introduces the Semi-implicit Fourier method 

\subsection{Micromechanics method of elastic and electric equations}
This section describes hot to use micromechanics to solve the elastic and electric static equations using mixed boundary conditions

\subsection{Iterative methods for inhomogeneous system}
This section describes the iterative method for inhomogeneous material systems 

\subsection{Step correction for taking derivatives}
This section describes the step correction trick to get numerical derivative of a Heaviside function 

\section{Examples}
\subsection{Phase transition of BaTiO3 singe crystal}
\subsection{Domain structures of PbTiO3 single crystal}
\subsection{Domain switching and hysteresis loop of PbTiO3 single crystal}
\subsection{Domain wall structures of BaTiO3}
\subsection{Domain structures and phase diagram of epitaxial BaTiO3 films}
\subsection{AFM tip-induced domain switching in (111)-BiFeO3 thin films}
\subsection{Mechanical switching in BaTiO3 via flexoelectric effect}
\subsection{Domain wall structure of BiFeO3 and oxygen octahedral rotation}

\section{Sources of errors}
\section{FAQ}

\end{document}
